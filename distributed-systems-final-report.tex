\documentclass{scrartcl}
\usepackage[utf8]{inputenc}
\usepackage{hyperref}
\usepackage{url}
\usepackage{natbib}
\usepackage{graphicx}
\usepackage{cleveref} % this must be the last package to be loaded

\newcommand{\emailaddr}[1]{\href{mailto:#1}{\texttt{#1}}}

\title{\LARGE
    Final Report\\
    OPC UA Industrial Plant HUB
}
\subtitle{Final Report for the Distributed Systems Course [A.A 2025-2026]}

% Consider watching:
% https://www.youtube.com/watch?v=ihxSUsJB_14
% https://www.youtube.com/watch?v=XTFWaV55uDo

\author{
    Federico Capitanio\\ \emailaddr{federico.capitanio@studio.unibo.it}
}

\date{\today}

\begin{document}

\maketitle

\begin{abstract}
    This report illustrates \textbf{OPCUA Industrial Plant Hub}, a software developed
    to demonstrate the fundamental principles of Distributed Systems by integrating them with the world
    of industrial automation.

    In the world of industrial automation, the issue of data collection from machines and machine lines is becoming increasingly important. Therefore, this project aims to provide a distributed data collection system for different production lines, which may also be distributed across the territory.

    The project is based on the use of IoT devices, which are connected to the production line and collect data on the production process. The data collected is then sent to a central server, which stores it and allows it to be analysed. The data can also be sent to the machines themselves, allowing them to be controlled and optimised.

    The underlying idea is that data is collected from the machine line using the \textbf{OPC UA} protocol, 
    widely used in industry, saved and then
    retrieved via \textbf{HTTP API}.

    Unlike traditional data collection architectures based on a single
    physical node that reads data, the architecture of this project consists of three
    HUB nodes that collect \textbf{OPC UA} data, synchronised via anti-entropy and gossip protocols,
    with an nginx load balancer that distributes client requests and implements
    data encryption via the \textbf{TLS} protocol.

    The project emphasises the \textbf{AP} behaviour of the \textbf{CAP Theorem}, i.e. it focuses
    on the aspects of \textbf{Partition Tolerance} and \textbf{High Availability}, thus ensuring
    that clients can always access data even in the presence of network partitions
    or individual node failures.
    The \textbf{Strong Consistency} concept is therefore not guaranteed, favouring
    data that is not updated to the latest possible value over a longer wait time that would result
    from the application of \textbf{Strong Consistency} itself.
    
    The system allows OPC UA sources to be added dynamically, by implementing
    an \textbf{on the fly} automation line connection mode, allowing 
    individual machines or entire lines to be connected and added without the system
    needing to be restarted.


\end{abstract}

\newpage


\section{Concept}\label{concept}

This project focuses on the development of a distributed system for collecting industrial data from different lines of automatic machines or plants, allowing for its centralisation.

\subsection{Product Type}\label{product-type}

\begin{itemize}
  \item A web interface that shows all the possible HTTP APIs and make HTTP requests available directly from the page itself;
  \item Setup scripts, one for Windows and one for Linux, to create ADMIN user and ADMIN password;
  \item OPC UA server simulator to instance various OPC UA nodes on each one that simulate real machine data;
  \item CLI commands to test the system itself, for instance switching off a node and restart it later.

\end{itemize}

\subsection{Use Case Description}\label{use-case-description}

\begin{itemize}
  \item \emph{What is the software doing?}
  The software initially instantiates three different nodes (there may be more). Each node connects to several OPC UA servers, browses all the variables present within the server itself, connects and starts collecting values every N seconds.
  The collected data is saved locally and replicated via anti-entropy with the other nodes performing the same task.
  The nodes also exchange messages with each other to know which nodes are actually active.
  An nginx server acting as a load balancer redirects HTTP requests coming from outside to the nodes that are alive and periodically tries to determine which of the previously down nodes are back online.
  
  \item \emph{when and how frequently do they interact with the system?}
  Users who will use the application will be anyone who makes an HTTP request to port 443 of the nginx server.
  
  \item \emph{when and how frequently do they interact with the system?}
  Since the software is not something like a game, users are defined as those who use the system itself and they can sent requests at any time as long as they don't exceed the Nginx brute force attack protection.
  
  \item \emph{how do they interact with the system? which devices are they using?}
  The user could interact with the system with every HTTP(s) client, so GUI client like Postman, CLI commands like \emph{curl} or even
  by building a front-end web page that use these application's API to show interactive web page to monitor machine performance or send command to the application itself.
  
\end{itemize}


\subsection{Why is distribution needed?}\label{why-distribution}

Distribution is a strictly necessary choice, as availability and reliability requirements are non-negotiable and constantly demanded in
the industrial sector. Below are the main reasons why distribution is necessary.

\begin{itemize}
  \item \emph{Fault Tolerance:}
  Each node (Ingestor) operates autonomously, collecting data from various OPC UA servers.
  Therefore, if any node fails, the system continues to operate with the remaining nodes.
  Furthermore, when this happens, anti-entropy mechanisms ensure automatic resynchronisation once
  the node (or nodes) becomes operational again.
  \item \emph{High Availability:}
  Industrial plants are designed to operate 24/7 most of the time, requiring
  constant monitoring systems with high availability.
  \item \emph{Scalability:}

\end{itemize}

\section{Requirements Elicitation and Analysis}\label{requirements-elicitation-and-analysys}


\bibliographystyle{plain}
\bibliography{references}

\end{document}
